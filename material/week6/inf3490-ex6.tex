\input{../header.tex}
\title{\vspace{-2cm}INF3490/INF4490 Exercises - Advanced Neural Networks}
\author{Eivind Samuelsen\input{../author_footnote.tex}}
\date{}

% Removing paragraph indents is sometimes useful:
\setlength\parindent{0pt}
% ==============================================================================

% ================================= DOCUMENT ===================================
\begin{document}
    \renewcommand\marginsymbol[1][0pt]{%
  \tabto*{0cm}\makebox[-1cm][c]{$\mathbb{P}$}\tabto*{\TabPrevPos}}

\maketitle
\input{../intro.tex}

\section{Perceptron activation functions}
Last week we used the activation function
\[
g(h) =
\begin{cases}
      1 & h > 0 \\
      0 & h \leq 0
   \end{cases}
\]
Why is this not used with backpropagation?

\section{Hidden layers}
What is the minimum number of hidden neuron layers needed in order to approximate an arbitrary continuous function, and why?

\section{Validation}
Why do we use a validation set?
Describe how the three different cross-validation methods presented in the lecture slides work, and what their advantages and disadvantages are.

\section{Multi Layer Perceptron \marginsymbol}
Implement the MLP shown below, and train it to correctly perform the XOR function.

\begin{figure}[H]
\begin{center}
\includegraphics[width=0.4\textwidth]{mlp.png}
\caption{MLP with one hidden layer and two hidden nodes.}
\label{fig:mlp}
\end{center}
\end{figure}

\section{Delta(error) function}
In the lecture slides the backpropagation deltas are first presented as
\[
\delta_k = (y_k - t_k)y_k(1-y_k)
\]
What does this tell us about the activation function in use?

\section{Natural language MLP}
You are to design an MLP that would learn to hyphenate words correctly.
You would have a dictionary that shows correct hyphenation examples for lots of words.
Think about the following:
What should the input to the neural net be?
\begin{itemize}
    \item How should this input be encoded to work well with the classifier?
    \item How is should the output be encoded?
    \item How many layers do you need?
    \item How many neurons should there be in each layer?
\end{itemize}
\input{../contact.tex}
\end{document}
% ==============================================================================
